%%%%%%%%%%%%%%
%% Run LaTeX on this file several times to get Table of Contents,
%% cross-references, and citations.

%% If you have font problems, you may edit the w-bookps.sty file
%% to customize the font names to match those on your system.

%% w-bksamp.tex. Current Version: Feb 16, 2012
%%%%%%%%%%%%%%%%%%%%%%%%%%%%%%%%%%%%%%%%%%%%%%%%%%%%%%%%%%%%%%%%
%
%  Sample file for
%  Wiley Book Style, Design No.: SD 001B, 7x10
%  Wiley Book Style, Design No.: SD 004B, 6x9
%
%
%  Prepared by Amy Hendrickson, TeXnology Inc.
%  http://www.texnology.com
%%%%%%%%%%%%%%%%%%%%%%%%%%%%%%%%%%%%%%%%%%%%%%%%%%%%%%%%%%%%%%%%

%%%%%%%%%%%%%
% 7x10
%\documentclass{wileySev}

% 6x9
\documentclass{wileysix}

\usepackage{graphicx}
\usepackage{listings}

\usepackage{color}
 
\definecolor{codegreen}{rgb}{0,0.6,0}
\definecolor{codegray}{rgb}{0.5,0.5,0.5}
\definecolor{codepurple}{rgb}{0.58,0,0.82}
\definecolor{backcolour}{rgb}{0.95,0.95,0.92}
 
\lstdefinestyle{mystyle}{
    backgroundcolor=\color{backcolour},   
    commentstyle=\color{codegreen},
    keywordstyle=\color{magenta},
    numberstyle=\tiny\color{codegray},
    stringstyle=\color{codepurple},
    basicstyle=\footnotesize,
    breakatwhitespace=false,         
    breaklines=true,                 
    captionpos=b,                    
    keepspaces=true,                 
    numbers=left,                    
    numbersep=5pt,                  
    showspaces=false,                
    showstringspaces=false,
    showtabs=false,                  
    tabsize=2,
    language=sh
}
 
\lstset{style=mystyle}

%%%%%%%
%% for times math: However, this package disables bold math (!)
%% \mathbf{x} will still work, but you will not have bold math
%% in section heads or chapter titles. If you don't use math
%% in those environments, mathptmx might be a good choice.

% \usepackage{mathptmx}

% For PostScript text
\usepackage{w-bookps}

%%%%%%%%%%%%%%%%%%%%%%%%%%%%%%%%%%%%%%%%%%%%%%%%%%%%%%%%%%%%%%%%
%% Other packages you might want to use:

% for chapter bibliography made with BibTeX
% \usepackage{chapterbib}

% for multiple indices
% \usepackage{multind}

% for answers to problems
% \usepackage{answers}

%%%%%%%%%%%%%%%%%%%%%%%%%%%%%%
%% Change options here if you want:
%%
%% How many levels of section head would you like numbered?
%% 0= no section numbers, 1= section, 2= subsection, 3= subsubsection
%%==>>
\setcounter{secnumdepth}{3}

%% How many levels of section head would you like to appear in the
%% Table of Contents?
%% 0= chapter titles, 1= section titles, 2= subsection titles, 
%% 3= subsubsection titles.
%%==>>
\setcounter{tocdepth}{2}

%% Cropmarks? good for final page makeup
%% \docropmarks

%%%%%%%%%%%%%%%%%%%%%%%%%%%%%%
%
% DRAFT
%
% Uncomment to get double spacing between lines, current date and time
% printed at bottom of page.
% \draft
% (If you want to keep tables from becoming double spaced also uncomment
% this):
% \renewcommand{\arraystretch}{0.6}
%%%%%%%%%%%%%%%%%%%%%%%%%%%%%%

%%%%%%% Demo of section head containing sample macro:
%% To get a macro to expand correctly in a section head, with upper and
%% lower case math, put the definition and set the box 
%% before \begin{document}, so that when it appears in the 
%% table of contents it will also work:

\newcommand{\VT}[1]{\ensuremath{{V_{T#1}}}}

%% use a box to expand the macro before we put it into the section head:

\newbox\sectsavebox
\setbox\sectsavebox=\hbox{\boldmath\VT{xyz}}

%%%%%%%%%%%%%%%%% End Demo


\begin{document}


\booktitle{Data Mining}
\subtitle{Pembahasan dan Kasus}

\authors{Nisa Hanum Harani\\
\affil{Informatics Research Center}
%Floyd J. Fowler, Jr.\\
%\affil{University of New Mexico}
}

\offprintinfo{Data Mining, First Edition}{Nisa Hanum Harani}

%% Can use \\ if title, and edition are too wide, ie,
%% \offprintinfo{Survey Methodology,\\ Second Edition}{Robert M. Groves}

%%%%%%%%%%%%%%%%%%%%%%%%%%%%%%
%% 
\halftitlepage

\titlepage


\begin{copyrightpage}{2019}
%Survey Methodology / Robert M. Groves . . . [et al.].
%\       p. cm.---(Wiley series in survey methodology)
%\    ``Wiley-Interscience."
%\    Includes bibliographical references and index.
%\    ISBN 0-471-48348-6 (pbk.)
%\    1. Surveys---Methodology.  2. Social 
%\  sciences---Research---Statistical methods.  I. Groves, Robert M.  II. %
%Series.\\
%
%HA31.2.S873 2007
%001.4'33---dc22                                             2004044064
\end{copyrightpage}

\dedication{}

\begin{contributors}
\name{Nisa Hanum Harani,} Informatics Research Center., Politeknik Pos Indonesia, Bandung,
Indonesia



\end{contributors}

\contentsinbrief
\tableofcontents
\listoffigures
\listoftables
\lstlistoflistings


\begin{foreword}
Sepatah kata dari Kaprodi, Kabag Kemahasiswaan dan Mahasiswa
\end{foreword}

\begin{preface}
Mempelajari Data Mining Dari Segi Permasalahan Yang Ada Di Lapangan.

\prefaceauthor{N. H. Harani}
\where{Bandung, Jawa Barat\\
Februari, 2019}
\end{preface}


\begin{acknowledgments}
Terima kasih atas semua masukan dari para mahasiswa agar bisa membuat buku ini 
lebih baik dan lebih mudah dimengerti.

Terima kasih ini juga ditujukan khusus untuk team IRC yang 
telah fokus untuk belajar dan memahami bagaimana buku ini mendampingi proses 
Intership.
\authorinitials{R. M. A.}
\end{acknowledgments}

\begin{acronyms}
\acro{ACGIH}{American Conference of Governmental Industrial Hygienists}
\acro{AEC}{Atomic Energy Commission}
\acro{OSHA}{Occupational Health and Safety Commission}
\acro{SAMA}{Scientific Apparatus Makers Association}
\end{acronyms}

\begin{glossary}
\term{git}Merupakan manajemen sumber kode yang dibuat oleh linus torvald.

\term{bash}Merupakan bahasa sistem operasi berbasiskan *NIX.

\term{linux}Sistem operasi berbasis sumber kode terbuka yang dibuat oleh Linus Torvald
\end{glossary}

\begin{symbols}
\term{A}Amplitude

\term{\hbox{\&}}Propositional logic symbol 

\term{a}Filter Coefficient

\bigskip

\term{\mathcal{B}}Number of Beats
\end{symbols}

\begin{introduction}

%% optional, but if you want to list author:

\introauthor{Nisa Hanum Harani, S.Kom., M.T.}
{Informatics Research Center\\
Bandung, Jawa Barat, Indonesia}

Pada era disruptif  \index{disruptif}\index{disruptif!modern} 
saat ini. git merupakan sebuah kebutuhan dalam sebuah organisasi pengembangan perangkat lunak.
Buku ini diharapkan bisa menjadi penghantar para programmer, analis, IT Operation dan Project Manajer.
Dalam melakukan implementasi git pada diri dan organisasinya.

Rumusnya cuman sebagai contoh aja biar keren\cite{awangga2018sampeu}.

\begin{equation}
ABC {\cal DEF} \alpha\beta\Gamma\Delta\sum^{abc}_{def}
\end{equation}

\end{introduction}

%%%%%%%%%%%%%%%%%%Isi Buku_

\chapter{Judul Bagian Pertama}
\section{Perintah Navigasi}
Perintah navigasi direktori


\chapter{Judul Bagian Kedua}
\section{Perintah Navigasi}
Perintah navigasi direktori




\bibliographystyle{IEEEtran} 
%\def\bibfont{\normalsize}
\bibliography{references}


%%%%%%%%%%%%%%%
%%  The default LaTeX Index
%%  Don't need to add any commands before \begin{document}
\printindex

%%%% Making an index
%% 
%% 1. Make index entries, don't leave any spaces so that they
%% will be sorted correctly.
%% 
%% \index{term}
%% \index{term!subterm}
%% \index{term!subterm!subsubterm}
%% 
%% 2. Run LaTeX several times to produce <filename>.idx
%% 
%% 3. On command line, type  makeindx <filename> which
%% will produce <filename>.ind 
%% 
%% 4. Type \printindex to make the index appear in your book.
%% 
%% 5. If you would like to edit <filename>.ind 
%% you may do so. See docs.pdf for more information.
%% 
%%%%%%%%%%%%%%%%%%%%%%%%%%%%%%

%%%%%%%%%%%%%% Making Multiple Indices %%%%%%%%%%%%%%%%
%% 1. 
%% \usepackage{multind}
%% \makeindex{book}
%% \makeindex{authors}
%% \begin{document}
%% 
%% 2.
%% % add index terms to your book, ie,
%% \index{book}{A term to go to the topic index}
%% \index{authors}{Put this author in the author index}
%% 
%% \index{book}{Cows}
%% \index{book}{Cows!Jersey}
%% \index{book}{Cows!Jersey!Brown}
%% 
%% \index{author}{Douglas Adams}
%% \index{author}{Boethius}
%% \index{author}{Mark Twain}
%% 
%% 3. On command line type 
%% makeindex topic 
%% makeindex authors
%% 
%% 4.
%% this is a Wiley command to make the indices print:
%% \multiprintindex{book}{Topic index}
%% \multiprintindex{authors}{Author index}

\end{document}

